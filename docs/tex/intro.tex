\chapter*{ВВЕДЕНИЕ}
\addcontentsline{toc}{chapter}{ВВЕДЕНИЕ}

Компьютерная графика -- это область информационных технологий, которая занимается созданием, обработкой, анализом и визуализацией графических объектов и изображений с использованием компьютерных методов и алгоритмов. 
Она имеет широкий спектр применений, включая развлечения, образование, научные исследования, медицинскую диагностику, архитектурное проектирование, визуализацию данных и многое другое.

Основные задачи, которые решает компьютерная графика.
\begin{enumerate}[label={\arabic*)}]
	\item Создание и редактирование изображений. 
    Компьютерная графика позволяет создавать и редактировать графические объекты и изображения на компьютере. 
    Это включает в себя рисование, моделирование форм и текстур, создание анимаций и спецэффектов.
    \item Визуализация данных. 
    В компьютерной графике данные могут быть представлены в визуальной форме, что делает их более понятными и удобными для анализа. 
    Например, диаграммы, графики и трехмерная визуализация данных используются для отображения статистических и научных результатов.
    \item 3D-моделирование и анимация. 
    Компьютерная графика позволяет создавать трехмерные модели объектов и сцен, а также анимировать их движение и поведение. 
    Это широко используется в фильмах, видеоиграх, архитектурном проектировании и медицинской визуализации.
\end{enumerate}

Общим для всех этих задач является использование математических алгоритмов, программного обеспечения и аппаратных средств для создания, модификации и визуализации графических данных с целью достижения определенных визуальных и функциональных результатов.

Для визуализации трехмерных сцен на конкретной аппаратной платформе часто требуется разрабатывать индивидуальные программные решения, учитывающие особенности данного обеспечения.

Цель данной работы -- спроектировать программно-аппаратный комплекс для построения моделей трехмерных объектов с использованием микроконтроллера.

Чтобы достичь поставленной цели, требуется решить следующие задачи: 
\begin{itemize}
    \item описать структуру трехмерной сцены, включая объекты, из которых состоит сцена, и определить формат задания исходных данных;
    \item выбрать наиболее подходящий из существующих алгоритмов трехмерной графики, позволяющих синтезировать изображение трехмерной сцены;
    \item реализовать выбранные алгоритмы построения трехмерной сцены;
    \item исследовать возможности микроконтроллеров в области визуализации трехмерной графики.
\end{itemize}
